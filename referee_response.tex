\documentclass[a4paper,11pt]{article}
\pdfoutput=1 % if your are submitting a pdflatex (i.e. if you have
% images in pdf, png or jpg format)

\usepackage{jcappub} % for details on the use of the package, please
% see the JCAP-author-manual

\usepackage[T1]{fontenc} % if needed
\usepackage[utf8]{inputenc}

%opening
\title{Reply to Referee of the paper Needlet estimation of cross-correlation between CMB lensing maps and LSS \texttt{JCAP\_047P\_0716} }

\begin{document}

\maketitle
\flushbottom

\section{}
First of all, we would like to thank you for your careful and thorough reading of the manuscript, as well as for your very constructive remarks, to which we respond as follows:
\begin{enumerate}
\item{ We agree on all the points you raised concerning comparative optimality of the estimators and the effective improvement that needlets can offer. We have modified the discussion both in the introduction and in the conclusions along the lines you suggested.}

\item{  We also agree that $S/N$ ratios are difficult to compare, because of the different bin size in the case of harmonic and needlet estimators. Indeed, we had mentioned this issue already in the previous version, but we have now discussed it much more clearly, especially in the conclusions. Moreover, we have expanded the discussion by adding Table~1 that reports the \textit{total} $S/N$ calculation comparison between the two approaches as suggested.}
  
\item{As you noted, there does not seem to be any intrinsic advantage when using either backward or forward modelling. We inserted some discussion about this point in Section 3.}

\item{ Done.}

\item{  Done.}

\item{  Done.}

\item{  Yes, we apply the $f_{\rm sky}$ correction given by eq.~3.4 (i.e. we divide the reconstructed psuedo spectra by $f_{\rm sky}$) when comparing input theory lines with extracted spectra. This is now clearly explained in the caption of Fig.~3.}

\item{  No, it is defined as MSE $= \langle (\hat{\beta}_j - \beta_j^{\rm th})^2 \rangle_{\rm MC}$ as explained in the text now.}

\item{We agree on that however, as written in the caption of Fig.~7, note that the ratio is evaluated for the extracted needlet spectra without considering MASTER-like corrections, i.e. exploiting the purple points shown in Fig.~3. In order to perform a better comparison between the two estimators we have inserted the results of the suggested calculation in Table~2.}

\item{As discussed above, a more quantitative assessment of the $S/N$ scaling  with $f_{\rm sky}$ can be found in the new Table 2.}

\item{We cut down some emphasis and replaced "extremely" with "rather" in the light of the performance comparison results. Bearing in mind that a precise comparison between the two approaches is not trivial, we have checked that the needlet performance in the H-ATLAS case can be better than the harmonic one.}

\end{enumerate}

\end{document}
